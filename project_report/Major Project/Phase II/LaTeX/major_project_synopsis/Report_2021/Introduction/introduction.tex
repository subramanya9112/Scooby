
%\chapter*{}
\section*{Introduction}
\graphicspath{{Introduction/IntroductionFigs/EPS/}{Introduction/IntroductionFigs/}}
The 21st century has been an era of colossal technological advances. Digital literacy is also one of them. It involves digital reading and writing techniques across multiple media forms. So there is an extensive increase in the use of electronic gadgets in this era. As a result, people are prone to various eye diseases. So, detecting them at an early stage is of utmost importance. 
Electroencephalography(EEG) is a test that detects abnormalities in the brain waves or the electrical activity of the brain. An EEG Sensor can be used to detect all those brain waves. Among those detected brain waves, some waves will be associated with the blink action. Those wave frequencies need to be separated and finally, we must calculate the blink frequency. 
Blink frequency helps detect many eye disorders. Detecting an eye disease at a very early stage will help in preventing many adverse effects. Hence, Blink analysis could be a very good contribution to society in detecting various eye diseases.
\section*{Objectives}
\begin{enumerate}
\item To find a suitable sensor and compatible device to detect the eye blinks
\item Collection of blink data required for blink analysis and making sure the data collected will be from both healthy patients and ones who are diagnosed with eye diseases.
\item To analyze the data collected from the selected device and extract blink-related information in real-time.
\item To display results obtained from blink analysis.
\end{enumerate}

% ----------------------------------------------------------------------
